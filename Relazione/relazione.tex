\documentclass{article}
\usepackage{titling}

\title{Relazione progetto Programmazione ad Oggetti}
\author{Alberto Canavese\\Numero matricola: 2076423}
\date{A.A. 2023/24}

\begin{document}
    % Pagina titolo
    \pretitle{\begin{center}\LARGE}
    \posttitle{\end{center}}
    \preauthor{\begin{center}\large \lineskip 0.5em}
    \postauthor{\end{center}}
    \predate{\begin{center}\large}
    \postdate{\end{center}}
    \maketitle

    % Pagina introduzione
    \newpage
    \section{Introduzione}
    \textbf{Sensor Hub} è un'applicazione realizzata con Qt Creator che consente di gestire tre tipi
    diversi di sensori virtuali.
    
    Il programma permette di creare, modificare, cancellare e ricercare i sensori tramite un'interfaccia
     grafica intuitiva.
        
    Vi sono due modi per dichiarare un sensore:
    \begin{itemize}
        \item \textbf{Tramite interfaccia}: Creando da 0 un sensore con i dati inseriti dall'utente tramite un'opportuna finestra di dialogo.
        \item \textbf{Tramite file}: Importando, tramite finstra di dialogo, un file txt contenente i dati del sensore.
    \end{itemize} 
    La funzionalità principale di \textbf{Sensor Hub} è la possibilità di avviare una simulazione di lettura dati per un qualsiasi sensore selezionato. 
    
    La simulazione genera una serie di dati, coerenti con il tipo di sensore, che vengono visualizzate graficamente utilizzando la classe \textbf{Qt Charts}.

    Tutti i dati di simulazione e tutti i dettagli di ogni sensore vengono salvati nel medesimo file di testo, in modo da supportare la persistenza dei dati. 

    Infine, è anche possibile esportare tutti i sopracitati dati sotto forma di file \textbf{.txt} tramite un context menu, il quale permette di aprire una schermata
    di dialogo utile per selezionare il percorso di destinazione del file generato dal programma.  

    \section{Descrizione del modello}
\end{document}
